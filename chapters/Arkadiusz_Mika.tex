\section{Arkadiusz Mika} 
\label{sec:amika}

\bigskip

\begin{center}
    {\Large \bf Nie żądam zapłaty za mocarność ani za mą przystojność.}
\end{center}

\begin{center}
    \begin{figure} [htbp]
        \centering
        \includegraphics[scale=0.25]{pictures/panda1.jpg} 
        \caption{\large Mój odwieczny wróg...Schody.}
        \label{fig:panda1}
    \end{figure}
\end{center}

\begin {flushleft}
    {\bf Tai Lung:} {\large Nie możesz mnie pokonać! Jesteś… jesteś tylko tłustym… słabym… pandą!} \par
    {\bf Po:} {\large Nie jestem tłustym, słabym pandą. Jestem tłustym i bardzo silnym pandą} [Po podnosi mały palec] \par
    {\bf Tai Lung:} {\large Technika Palca Zagłady!} \par
    {\bf Po:} {\large O… znasz tę technikę!} \par
    {\bf Tai Lung:} {\large Blefujesz… blefujesz! Shifu by Cię tego nie nauczył!} \par
    {\bf Po:} {\large Fakt! Sam się nauczyłem – szypyciu. }\par
\end {flushleft}

Najwięksi mocarze ~\ref{itemize:amika_list1}:
\begin{enumerate}
  \item Po
  \item Oogway
  \item Shifu
  \label{itemize:amika_list1}
\end{enumerate}

\bigskip

Niemocarze ~\ref{itemize:amika_list2}:
\begin{itemize}
  \item[*] Tai Lung
  \item[*] Lord Shen
  \item[*] Jednooki wilk
  \label{itemize:amika_list2}
\end{itemize}

\begin{table}[h]
\begin{tabular}{|c|c|ll}
\cline{1-2}
Cyfry & Cyfry v2 &    \\ \cline{1-2}
1 & 8   \\
2 & 9  \\ \cline{1-2}
\end{tabular}
\centering
\label{tab:amika_tab}
\caption{Simple table}
\end{table}

W szczególności, dla każdej liczby rzeczywistej {\it a} prawdziwa jest równość:
 \[\sqrt{a^2} = |a|\]